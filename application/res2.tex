\documentclass[letterpaper, 10pt]{article}
\topmargin-2.0cm

\usepackage{fancyhdr}
\usepackage{lastpage}
\usepackage{xspace}
\usepackage[usenames,dvipsnames,svgnames,table]{xcolor}

% Color Information from -
% http://www-h.eng.cam.ac.uk/help/tpl/textprocessing/latex_advanced/node13.html

\advance\oddsidemargin-1in
%\advance\evensidemargin-0.65in
\textheight9.2in
\textwidth6.9in
\newcommand\bb[1]{\mbox{\em #1}}
\def\baselinestretch{1.05}

\newcommand{\hsp}{\hspace*{\parindent}}
\definecolor{gray}{rgb}{0.4,0.4,0.4}

\begin{document}
\thispagestyle{fancy}

% Leave Left and Right Header empty.
\lhead{}
\rhead{}

\renewcommand{\headrulewidth}{0pt} \renewcommand{\footrulewidth}{0pt}
\fancyfoot[C]{\footnotesize
\textcolor{gray}{http://www.cs.columbia.edu/$\sim$jikk/application}}

\pagestyle{fancy}
\lhead{\textcolor{gray}{\it Kangkook Jee}}
\rhead{\textcolor{gray}{\thepage /\pageref{LastPage}}}

%Macros of my own
\def\libdft{libdft\xspace}
\def\TFAFull{Taint Flow Algebra\xspace}
\def\TFA{TFA\xspace}
\def\SR{ShadowReplica\xspace}
\def\ie{i.e.,\xspace}

\newif \ifcomments
%\commentstrue

\ifcomments
\newcommand{\jikk}[1]{{---\textcolor{red}{#1}---}}
\else
\newcommand{\jikk}[1]{}
\fi
%

%\vspace*{0.1cm}
\begin{center}
{\LARGE \bf RESEARCH STATEMENT}\\
\vspace*{0.1cm}
{\normalsize Kangkook Jee (jikk@cs.columbia.edu)}
\end{center}

The task of {\bf building secure and reliable computer system} is getting
harder most notably for failing to propose a security system that addresses the
following properties simultaneously.

\begin{itemize}

        \item {\it Effectiveness} in detecting and preventing software failures
                caused either by active exploitation of malicious attackers or
                unintended disclosure of developer bugs.

        \item {\it Efficiency} by having minimal amount of resource
                overhead~(e.g., power, response time) to make the security
                solution adoptable to the production environment.

        \item {\it Coverage} in providing end-to-end protection. For instance,
                security system should protect the entire software execution
                stack supporting COTS binaries with no source access.

\end{itemize}

The rapid evolution of computer systems only exacerbates the problem as it
constantly re-defines our notion about the computing environment. Many industry
vendors as well as research communities have driven the innovations in the
research area of reliable and secure system. This is the space in which work
is. \\


My main research contributions have been concretized over the past a few years
as I developed and improved well-known security architecture of {\bf Data Flow
Tracking (DFT)} that supports COTS binaries. To address the issue of high
overhead inherent to the systems of the same kind, I leveraged a number of
recent research innovations from different CS fields. Specifically, I have been
interested in {\it i)} combining static and dynamic analysis approaches to
adopt operating system and programming language theories {\it ii)} facilitating
multi-core architecture to implement parallelized analysis. Although my
research have focused on a specific system of DFT, I believe that the insights
that I learned from it can benefit the area of system security in general since
DFT shares many architectural characteristics with other security systems. \\

In addition to making improvements to a dynamic security system, I have a
broader interest in research topics related to secure and reliable software
system. Firstly, it is crucial to improve static analysis techniques such as
{\bf white-box and black-box software fuzzing} to assist developers. Secondly,
we also need to improve the {\bf educational system} to have our students
better understandings about various security issues. In my view, the goal of
building secure computing environment can be attained not by any single
research direction, but by holistic consideration of a number of different
approaches including aforementioned ones. \\

My long-term research will be about the design and implementation of the
security architecture, which would impact everyone’s computing life by
providing trusted computing service at the expense of the negligible runtime
slowdown. The approach that I am currently focusing on is to take advantage of
the static analysis techniques to improve the dynamic defense measures. The
other way around, executions from dynamic component would guide the static
counterpart to achieve better coverage and accuracy. Complementing one another,
I am confident that we are getting closer to the goal of building secure and
reliable system effective in countering wide range of attack vectors (often
unpredictable) only requiring minimal amount of overhead. \\

From my research results thus far, I believe that I have already opened up the
promising perspectives in this research direction. 

\end{document}

