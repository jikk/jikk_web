
\documentclass[letterpaper, 10pt]{article}
\topmargin-2.0cm

\usepackage{fancyhdr}
\usepackage{hyperref}
\usepackage{lastpage}
\usepackage[dvips]{color}
\usepackage{graphicx}
\usepackage[usenames,dvipsnames,svgnames,table]{xcolor}
\usepackage{xspace}

% Color Information from -
% http://www-h.eng.cam.ac.uk/help/tpl/textprocessing/latex_advanced/node13.html

\advance\oddsidemargin-1in
\advance\evensidemargin-1.5cm
\textheight9.2in
\textwidth6.75in
\newcommand\bb[1]{\mbox{\em #1}}
\def\baselinestretch{1.05}

\newif \ifcomments
%\commentstrue

\ifcomments
\newcommand{\jikk}[1]{{---\textcolor{red}{#1}---}}
\else
\newcommand{\jikk}[1]{}
\fi

\def\ie{i.e.,\xspace}

\newcommand{\hsp}{\hspace*{\parindent}}
\definecolor{gray}{rgb}{0.4,0.4,0.4}

\begin{document}
\thispagestyle{fancy}

% Leave Left and Right Header empty.
\lhead{}
\rhead{}

\renewcommand{\headrulewidth}{0pt} 
\renewcommand{\footrulewidth}{0pt} 

\fancyfoot[C]{\footnotesize
        \textcolor{gray}{http://www.cs.columbia.edu/$\sim$jikk/application}} 

\pagestyle{fancy}
\lhead{\textcolor{gray}{\it Kangkook Jee}}
\rhead{\textcolor{gray}{\thepage /\pageref{LastPage}}}

% This kind of makes 10pt to 9 pt.
\begin{small}

%\vspace*{0.1cm}
\begin{center} {\LARGE \bf TEACHING STATEMENT}\\ \vspace*{0.1cm} {\normalsize
Kangkook Jee (jikk@cs.columbia.edu)} \end{center}
% Begin with my teaching philosophy.
University education should provide opportunities for students to cultivate
their potential to succeed in their future careers as competent computer
science professionals.  
%
In teaching CS courses, I focus on helping students prepared with a principled
understanding about fundamental CS topics to adopt best practices both from
academia and industry.
%
To achieve the goal, I lay emphasis on establishing an active communication
channel that connects the educator and students.

% I enjoy human interactions in the course of teaching.
\vspace{-2pt}
\subsubsection*{In class teaching experience}

During my PhD, I taught Columbia University's COMS3103-3: Programming Language
Python\footnote{\url{http://www.cs.columbia.edu/~jikk/teaching/3101-3/index.html}}.
%
% The course begin with 22 students and 14 students completed the course. For
% failing to set appropriate level for assignment.
%
Although it was my first university level teaching experience, I deeply enjoyed
the whole process of building a course from scratch, deliver prepared
materials, making interactions, and evaluating student progresses.
% 
The course is largely composed of two distinct phases. The first part is about
language fundamentals while the second part focuses on specific topics relevant
to the students' interest. Establishing the latter was relatively challenging
since the class comprised undergraduates, MS and PhD students from diverse
disciplines of CS, economics, philosophy, English literature and so on.
%
Surveying to know about student's expectation and their background at the
beginning of the semester, I put an effort in maximizing interactions with
students in and out of the classroom.\jikk{as well as providing
definite/direct/concrete help in solving each student's problem.}
% 
For the course, students were to fulfill four homework assignments and a class
project to complete the course. 
%
The assignments were designed not only to assess the student's comprehension
with course materials, but also to familiarize non-major students with basic CS
concepts.
%
% basic data structures such as queue and stack as well as an algorithmic
% concept of TSP were used.
%
Given that students always can search for solutions for re-used problem sets,
writing new ones for each assignment required a substantial amount of time and
effort. 
%
Regarding a class project, students were to team up with 3 $\sim$ 4 members to
conduct to a project relevant to their research interest.
%
It was an exciting experience to see many project outputs being mature and
creative reaching out beyond my expectation.
%
% Some impressive example include search engine for Columbia university, NLP
% project that establishes character network extraction analyzing novel text
% done by English literature major.
%

Compensating a challenging task of doing a class, the course evaluation
result\footnote{\url{http://www.cs.columbia.edu/~jikk/teaching/3101-3/evaluation.html}}
of students were encouraging. In a nutshell, students rated 4.45/5 for
overall course quality and 4.64/5 for the amount they learned from the class.

\vspace{-2pt}
%CS as a moving target
\subsubsection*{Designing new courses} 

CS as an academic area and its many sub-fields are still young and vibrantly
evolving. Therefore, it is required to design and offer new courses that
reflect recent updates without losing the connections to fundamental theories
supporting the topic. From here, I will introduce two course ideas that I want
to teach in the near future. 

{\bf C++ and new standard~(C++11):} In the history of programming languages,
C++ occupies a unique position and it entered into an interesting new stage
with its new standard~(C++11) announced recently and a solid roadmap projecting
a decade ahead.
%Although many complain for its syntax being too complex and its features being
%not organized and only being grossly accumulated over time.  But, there is few
%alternative when we look for the programming language that supports the
%low-level access and close-to native performance in a scalable way.
%
%Based C language, reaching out to absorb high-level language concepts
%including OOP paradigm, C++ enters interesting new stage with its recent
%announcement for a new standard(C++11) and concrete roadmap projecting a
%decade ahead which are well appreciated by community.
%
By having a course about C++ and its new standard, we can expect students not
only to learn a language and its newly added features, but also to understand
how a low level language~(based on C) embraces high level language concepts. 

{\bf Instrumentation approaches from the source to the runtime:} 
% What else courses I'm capable of/interested in teaching.
Throughout my research career, I explored software instrumentation techniques
from different software layers which are deeply related to the steps we take to
compile a source into a binary and load it for execution. 
%
As {\it sources} first being compiled into {\it objects}, compilers expose their 
internal representations of AST and IR as an instrumentation target.
%
Link-time-optimization~(LTO) provide another instrumentation opportunity as a
linker takes {\it objects} to generate output {\it binary}. 
%
On execution time, we can consider tools of varying levels; \ie {\it program
loader} or {\it VM-hypervisor} of different types.
%
This one will be considered as  advanced compiler course focusing on delivering
concepts for program compilation and runtime execution as well as
instrumentation approaches to write various security/reliability monitors.

Besides the aforementioned topics, I am eager to teach courses of varying
levels of CS discipline. Interested and specialized teaching topics would
include programming language theory, operating systems, network and system
security.

\subsubsection*{Summary}

In my view, university educators should maximize the face-to-face
experience\jikk{face-time} with their students and design each course
reflecting students demand without loosing track of evolving
research and industry trend.
%
Based on encouraging result of my past teaching experience, I am looking
forward to have a chance to meet with students to share my experience and
knowledge. 

\end{small}
\end{document}

