
\documentclass[letterpaper, 10pt]{article}
\topmargin-2.0cm

\usepackage{fancyhdr}
\usepackage{hyperref}
\usepackage{lastpage}
\usepackage[dvips]{color}
\usepackage{graphicx}
\usepackage[usenames,dvipsnames,svgnames,table]{xcolor}
\usepackage{xspace}

% Color Information from -
% http://www-h.eng.cam.ac.uk/help/tpl/textprocessing/latex_advanced/node13.html

\advance\oddsidemargin-1in
\advance\evensidemargin-1.5cm
\textheight9.2in
\textwidth6.75in
\newcommand\bb[1]{\mbox{\em #1}}
\def\baselinestretch{1.05}

\newif \ifcomments
\commentstrue

\ifcomments
\newcommand{\jikk}[1]{{---\textcolor{red}{#1}---}}
\else
\newcommand{\jikk}[1]{}
\fi

\def\ie{i.e.,\xspace}

\newcommand{\hsp}{\hspace*{\parindent}}
\definecolor{gray}{rgb}{0.4,0.4,0.4}

\begin{document}
\thispagestyle{fancy}

% Leave Left and Right Header empty.
\lhead{}
\rhead{}

\renewcommand{\headrulewidth}{0pt} 
\renewcommand{\footrulewidth}{0pt} 

\fancyfoot[C]{\footnotesize
        \textcolor{gray}{http://www.cs.columbia.edu/$\sim$jikk/application}} 

\pagestyle{fancy}
\lhead{\textcolor{gray}{\it Kangkook Jee}}
\rhead{\textcolor{gray}{\thepage /\pageref{LastPage}}}

% This kind of makes 10pt to 9 pt.
\begin{small}

%\vspace*{0.1cm}
\begin{center} {\LARGE \bf TEACHING STATEMENT}\\ \vspace*{0.1cm} {\normalsize
Kangkook Jee (jikk@cs.columbia.edu)} \end{center}
% Begin with my teaching philosophy.
University education should provide opportunities for students to cultivate
their potentials to succeed in their future careers as a competent computer
science professionals.  
%
As an CS educator, I focus on preparing students with principled understanding
about fundamental/theoretical CS topics adopting best practices both from
academia and industry. Core to this process, courses should be built based on
inter-actions. 
\jikk{improve intro to match with conclusion.}


%I enjoy human interactions in the course of teaching.

\subsubsection*{In class teaching experiences}

During my PhD, I taught Columbia University's COMS3103-3: Programming Language
Python\footnote{\url{http://www.cs.columbia.edu/~jikk/teaching/3101-3/index.html}}.
%
% The course begin with 22 students and 14 students completed the course.  for
% failing to set appropriate level for assignment.
%
Although it was my first university level teaching experience, I deeply enjoyed 
for the whole process of building a course from scratch, deliver prepared
materials, making interactions with students, and evaluating their progresses.
% 
The course is largely composed of two distinct phases. The first one is about
language fundamentals while the second one focuses on specific topics/modules
relevant to student's interest. Establishing the latter was especially
challenging since the class comprised undergraduates, MS and PhD students from
diverse disciplines of CS, economics, philosophy, English literature and so on.
%
I began the semester surveying to know about students' expectation and the
background and I also attempted to maximize to face-to-face interaction with
each student throughout the semester.
% 
Students were to fulfill four homework assignments and a class project to
complete the course. 
%
Assignments were designed not only to assess student's comprehension about
course materials but also to introduce primitive CS concepts to non-major
students.
%
% basic data structures such as queue and stack as well as an algorithmic
% concept of TSP were used.
%
I also realized that it requires  substantial amount of time and effort to
write a new set of assignments each time given that students always can search
for solutions for known problem sets. 
%
Regarding a class project, students were to made teams of 3 $\sim$ 4 members to
conduct to a project relevant to their research interest.
%
It was an exciting experience to find out that many of project output were so
creative and reached out beyond my expectation.
%
% Some impressive example include search engine for Columbia university, NLP
% project that establishes character network extraction analyzing novel text
% done by English literature major.
%

Compensating a challenging task of doing a class from the beginning to the end,
the course evaluation
result\footnote{\url{http://www.cs.columbia.edu/~jikk/teaching/3101-3/evaluation.html}}
performed by student were encouraging. In a nutshell, students rated 4.45/5 for
overall course quality and 4.64/5 for amount they learned from the class.

%CS as a moving target
\subsubsection*{Designing new courses} 

CS as an academic area and its many sub-fields are still young and vibrantly
evolving. Therefore, it is required to timely design and offer new courses that
reflect recent trends/changes/updates without loosing the connections to
fundamental theories supporting the topic. \jikk{improve it} From here, I will
introduce brief descriptions for a couple of new course offerings.
%

{\bf C++ and new standard~(C++11):} In the history of programming languages, C++
occupies a unique position and it opens up interesting new stage with its
recent announcement for a new standard~(C++11) and a solid roadmap projecting a
decade ahead which are well appreciated by community.
%Although many complain for its syntax being too complex and its features being
%not organized and only being grossly accumulated over time.  But, there is few
%alternative when we look for the programming language that supports the
%low-level access and close-to native performance in a scalable way.
%
%Based C language, reaching out to absorb high-level language concepts
%including OOP paradigm, C++ enters interesting new stage with its recent
%announcement for a new standard(C++11) and concrete roadmap projecting a
%decade ahead which are well appreciated by community.
%
By having a course about C++ and its new standard, we can expect students not
only to lean a language and its newly added features but also to understand how
a low level language~(based on C) reaches out to have high level language
concepts in place. 

{\bf Instrumentation techniques from source to runtime:} 
% What else courses I'm capable of/interested in teaching.
Throughout my research career, I explored software instrumentation techniques
from different software stacks/substrates/layers which are deeply related to
steps we take to compile a source to build a binary and load it for execution. 
%
As {\it sources} first being compiled into {\it objects} , compilers expose its
internal representations of AST and IR as a instrumentation target.
%
Link-time-optimization~(LTO) also provides another instrumentation opportunity
at link time. % as we {\it link} objects to produce a binary. 
%
On execution, we can consider tools from different levels for the
instrumentations; \ie {\it program loader} or {\it VM-hypervisor} of different
types.
%
This course will be considered as a one that cover advanced compiler topics
focusing on delivering end-to-end concepts for program compilation and
execution model and instrumenation approach to build various system monitoring
purposes.

Besides aforementioned topics, I am eager to different courses from any level
of CS disiplines. Interested/specialized teaching topic would include courses
for programming language theory, operating systems, network and system
security.

\subsubsection*{Summary}

In my view, university educators should maximize face-to-face experience with
their students and build/design each course as a result of those interactions
reflecting students demand without loosing track of updated research and
industry trend.
%
Based on encouraging result from my past teaching experience, I am looking
forward to have a chance to meet with students and share my experience and
knowledge. 


\end{small}
\end{document}

