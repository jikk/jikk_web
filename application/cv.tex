%%%%%%%%%%%%%%%%%%%%%%%%%%%%%%%%%%%%%%%%%%%%%%%%%%%%%%%%%%%%%%%%%%%%%%%%%%%%%%
%                             Curriculum Vitae                               %
%%%%%%%%%%%%%%%%%%%%%%%%%%%%%%%%%%%%%%%%%%%%%%%%%%%%%%%%%%%%%%%%%%%%%%%%%%%%%%
\documentclass[10pt,a4]{article}
\topmargin-2.0cm
\advance\oddsidemargin-1.2cm
\advance\evensidemargin-1.2cm
\textheight9.22in
\textwidth6.4in
\newcommand\bb[1]{\mbox{\em #1}}
%\def\baselinestretch{1.25}
\def\baselinestretch{1.0}

\usepackage{multicol}
% The use of the times package forces the use of the type-1 times
% roman font, but the times roman font does not look nice.
% Besides the times roman font still does not print correctly on
% the dopy printer.
%\usepackage{times}

\usepackage{fancyhdr}
\usepackage[dvips]{color}

\newcounter{myEnumCounter}
\newcounter{mySaveCounter}
\renewenvironment{enumerate}{%
  \begin{list}{\arabic{myEnumCounter}.}{\usecounter{myEnumCounter}%
  \setcounter{myEnumCounter}{\value{mySaveCounter}}}
  }{%
  \setcounter{mySaveCounter}{\value{myEnumCounter}}\end{list}%
}
\newcommand\myEnumReset{\setcounter{mySaveCounter}{0}}

% The old enumerate environment is rewritten, so you need no special command to
% start continuing counting. With the command \myEnumReset you can Reset the couter
% at any place in the text.

% http://www.educat.hu-berlin.de/~voss/lyx/list/enum.phtml

\definecolor{gray}{rgb}{0.4,0.4,0.4}

\begin{document}

%\thispagestyle{empty}
%\pagestyle{plain}

\thispagestyle{fancy}
%\pagenumbering{gobble}
%\fancyhead[location]{text}
% Leave Left and Right Header empty.
%\lhead{\textcolor{gray}{\it Sundar Iyer}}
%\rhead{\textcolor{gray}{\thepage/\totalpages{}}}
%\rhead{\thepage}
\renewcommand{\headrulewidth}{0pt}
\renewcommand{\footrulewidth}{0pt}
\fancyfoot[C]{\footnotesize \textcolor{gray}{}}
%A copy of this curriculum vitae, publications and 
%talk slides are available for download at
%http://www.stanford.edu/$\sim$sundaes/application}}


%\pagestyle{myheadings}
%\markboth{Sundar Iyer}{Sundar Iyer}

\vspace*{0.4cm}
\begin{center}
{\huge \bf Kangkook Jee}
\vspace*{0.25cm}
\end{center}

\begin{small}

%===================================
\begin{tabbing}
\=xxxxxxxx\=xxxxxxxx\=xxxxxxxx\=\kill
\begin{tabular*}{\linewidth}{l@{\extracolsep{\fill}}r}

1214 Amsterdam Ave. Mailcode: 0401  & Phone: (212) 939-7054 \\
New York, NY 10027 &  Email: jikk@cs.columbia.edu\\
http://www.cs.columbia.edu/$\sim$jikk & Alt: kangkook.jee@gmail.com \\
\end{tabular*}
\end{tabbing}

\vspace*{0.2cm}

%==========================================

\subsection*{PARTICULARS}

\hrule
\vspace{0.2cm}
%%%%%%%%%%%%%%%%%%%%%%%%%%%%%%%

\subsubsection*{EDUCATION}

\begin{tabbing}
xxxx\=xxxxxxxx\=xxxxxxxx\=xxxxxxxx\=\kill

\>\begin{tabular*}{0.9\linewidth}{l@{\extracolsep{\fill}}r}
Columbia University & New York, NY \\
Ph.D. in Computer Science  &  {\it 2008-, Expected in May 2014}\\
 & \\

Columbia University & New York, NY \\
M.S. in Computer Science & {\it May 2008}\\
 & \\

Korea University, & Seoul, South Korea \\
B.Sc. majoring Mathematics minoring Computer Science & {\it  Mar 2001}
\end{tabular*}
\end{tabbing}

\subsubsection*{CURRENT STATUS}
\begin{list}{}{}
\item Citizen of South Korea.
\end{list}

\subsubsection*{RESEARCH INTERESTS}
%\hrule
%\vspace{0.2cm}

\begin{list}{}{}
% Do I have to specifically mention DFT or Symbolic execution?
\item My research interest lies in protecting software systems from either
  malcious attempts of attacker outside or unintended accidents due to
  programmer mistakes. To implement systems that enhance defense capabilities of
  software systems, I leverage tools available from different software layers of
  VM hypervisor, OS kernel, program compiler.
\end{list}

\subsection*{WORK EXPERIENCE}
\hrule
\vspace{0.2cm}
\begin{itemize}
\item {\bf IBM Korea},  Mar 2001 - Aug. 2006.  \\
  {\bf Advanced software engineer} (July 2004 - Aug. 2006): Provided second line
  support for AIX operating system, IBM Java Virtual Machine (JVM), and high
  availability solutions. \\
  {\bf Software specialist} (Mar 2001 - Jun 2004): Supported AIX operating system
  installed in Samsung Electronics. \\
  {\bf Awarded IBM stock options,} 600 stocks: (Nov 2004)\\
%
\item {\bf Republic of Korea Army}, Jan. 1997 - Mar. 1999. \\
{\bf 18th Medical Command, 8th U.S. Army}: Technical staff for Information
management office, 121 general hospital Yongsan army base, South Korea as a part
of mandatory military service.\\

\end{itemize}

%\vspace{0.1cm}
\subsection*{RESEARCH EXPERIENCE}
\hrule
\vspace{0.2cm}
\begin{itemize}
\item {\bf Master Student Research Assistant, Columbia University}, Sep 2007 - May 2008. \\
%
  {\bf Application Community}, (Jan 2007 - Aug 2009): Designed and implemented
  security architecture that segments and distributes defense and protection
  responsibility across the large number of hosts connected via the Internet
  leveraing software mono-cultures. The insight behind this approach to
  alleviates the issue of high overhead which is unavoidable to the most of
  existing dyanamic security solutions while maintaining the rate of the
  detection statastical threshold that we set.
% TODO: some closing setence should follow.

\vspace{0.1cm}
\item  {\bf Research Assistant, Columbia University}, June 2007 - Current. \\
%
%   {\bf MINESTRONE: IDENTIFYING AND CONTAINING SOFTWARE VULNERABILITIES} (
%   IARPA project, 08/2010 - 07/2014): MINESTRONE is a novel architecture that integrates
%   static analysis, dynamic confinement, and code diversification techniques to
%   enable the identification, mitigation and containment of a large class of
%   software vulnerabilities. Our techniques will protect new software, as well as
%   already deployed (legacy) software by transparently inserting extensive
%   security instrumentation. \\
%   %
%   {\bf SPARCHS: Symbiotic, Polymorphic, Autotomic, Resilient, Clean-slate, Host
%     Security} (DARPA CRASH project, 10/2010 - 09/2014): The SPARCHS project is
%   considering a new computer systems design methodology that considers security
%   as a first-order design requirement at all levels, starting from hardware, in
%   addition to the usual design requirements such as programmability, usability,
%   speed, and power/energy efficiency.
%   
  {\bf libdft}, (Jan 2010 - Dec 2011): Dynamic data flow tracking (DFT) deals
  with tagging and tracking data of interest as they propagate during program
  execution.  We presented libdft, a dynamic DFT framework that unlike previous
  work is at once fast, reusable, and works with commodity software and
  hardware. libdft provides an API for building DFT-enabled tools that work on
  unmodified binaries, running on common operating systems and hardware, thus
  facilitating research and rapid prototyping.\\

  {\bf Taint Flow Algebra}, (Jan 2010 - Dec 2011): Our intuition is to separate
  the program logic from the corresponding tracking logic, extracting the
  semantics of the latter and abstracting them using a Taint Flow Algebra. We
  then apply optimization tech- niques to eliminate redundant tracking logic and
  min- imize interference with the target program. Our op- timizations are
  directly applicable to binary-only soft- ware and do not require any high
  level semantics. Fur- thermore, they do not require additional resources to
  im- prove performance, neither do they restrict or remove functionality.\\

  {\bf ShadowReplica}, (Jan 2012 - June 2013): ShadowReplica is a new and
  efficient approach for accelerating DFT and other shadow memory-based
  analyses, by decou- pling analysis from execution and utilizing spare CPU
  cores to run them in parallel. DFT is run in parallel by a second shadow
  thread that is spawned for each application thread, and the two communicate
  us- ing a shared data structure. We avoid the problems suffered by previous
  approaches, by introducing an off-line application analysis phase that
  utilizes both static and dynamic analysis method- ologies to generate
  optimized code for decoupling execution and implementing DFT, while it also
  minimizes the amount of infor- mation that needs to be communicated between
  the two threads.\\
% %
%   {\bf Comparison Study for Different Dynamic Binary Instrumentation(DBI)
%     Systems}, (Jan 2001 - Mar 2002): From a network designer's perspective a
%   shared memory router is ideal in that the packets are stored in a central
%   location and the memory bandwidth and space is shared across packets from all
%   ports. This sharing helps in conserving memory space and results in low cost
%   and low power routers. However, it is a widely held myth that such shared
%   memory routers are not scalable to higher speeds due to limitations imposed on
%   the speed of a single memory. I worked on the first demonstration and proof to
%   show how to build and scale the capacity of shared memory routers using
%   distributed memories.\\
%

  {\bf A tool for runtime detection of integer errors written in C/C++}, (Jun
  2012 - Jul 2002): In this project, we integrate static and dynamic data flow
  tracking(DFT) to integer error detection system to address this issue of false
  positive(FP) and false negative(FN) specifically. Using DFT, we define
  relevant (source, sink) pairs that would differentiate integer errors which
  lead to serious bugs that the attackers can leverage to exploit. One example
  of these pairs can be values come through untrusted input(e.g., socket input,
  keyboard inputs) and flow into arithmetic operations that triggers integer
  errors. The other example can be pairs of the location of integers errors
  flows into security sensitive places such as arguments for memory allocation
  functions. \\
\end{itemize}


%\vspace{0.1cm}
\subsection*{TEACHING EXPERIENCE}
\hrule
\vspace{0.2cm}

\begin{itemize}

\item {\bf Instructor.} COMS 3103-3: Programming Languages: Python , Fall 2013,
  Columbia University.\\
  14 on-campus students (4.25/5)

\item {\bf Teaching Assistant.} COMS 4701: Artificial Intelligence, Prof. Sal Stolfo
Spring 2013, Columbia University.

\item {\bf Teaching Assitant} COMS 3157-1: Advanced Programming, Prof Angelos
  D. Keromytis

\end{itemize}


% List of papers and publications and patents ?
%===========================================
%\vspace{0.1cm}
\subsection*{PUBLICATIONS}
\hrule
\vspace{0.2cm}

\subsubsection*{PAPERS}

\begin{enumerate}
    \item 
K. Jee, V. P. Kemerlis, A. D. Keromytis and G. Portokalidis.
``ShadowReplica: Efficient Parallelization of Dynamic Data Flow Tracking",
to appear in {\it Proceedings of ACM CCS}, Nov. 2013.

    \item 
V. P. Kemerlis, G. Portokalidis, K. Jee, and A. D. Keromytis.
``libdft: Practical Dynamic Data Flow Tracking for Commodity Systems", 
to appear in {\it Proceedings of ACM VEE}, Apr. 2012.

    \item 
K. Jee, G. Portokalidis, V. P. Kemerlis, S. Ghosh, D. I. August, and
A. D. Keromytis.
``A General Approachfor Effciently Accelerating Software-based Dynamic Data Flow
Tracking on Commodity Hardware ", to appear in {\it Proceedings of NDSS} Feb. 2013.

    \item 
K. Jee, S. Sidiroglou-Douskos, A. Stavrou, and A. D. Keromytis.
``An Adversarial Evaluation of Network Signaling and Control Mechanisms", 
to appear in {\it Proceedings of ICISC} Dec. 2010
\end{enumerate}

\subsubsection*{BOOK}
\begin{enumerate}
    \item
K. Hayashi, K. Ji, O. Lascu, H. Pienaar, S. Schreitmueller, T. Tarquinio, J. Thompson.
``AIX5L Practical performance and tuning guide", published by IBM Press books,
ISBN-0738491799 March 2005
\end{enumerate}
\myEnumReset

%\vspace{0.1cm}
\subsection*{TALKS}
\hrule
\vspace{0.2cm}

\subsubsection*{CONFERENCE TALKS}
\begin{enumerate}
    \item
       ``ShadowReplica: Efficient Parallelization of Dynamic Data Flow Tracking",
       {\it ACM CCS 2013}, Berlin, Germany, Nov. 2013. 

     \item 
       ``A General Approachfor Effciently Accelerating Software-based Dynamic
       Data Flow Tracking on Commodity Hardware", {\it NDSS 2012}, San Diego,
       California, Feb. 2012.

    \item
       ``An Adversarial Evaluation of Network Signaling and Control Mechanisms",
       {\it ICISC 2010}, Seoul, South Korea, Dec. 2010. 
\end{enumerate}

\subsubsection*{INDUSTRY/OTHER TALKS}
\begin{enumerate}
\item
  ``A General Approachfor Effciently Accelerating Software-based Dynamic Data
  Flow Tracking on Commodity Hardware", IBM Programming Language Day 2012,
  Jun. 2012, IBM Thomas J. Watson Research Center.
\end{enumerate}

\subsection*{LANGUAGES}
\hrule
\vspace{0.2cm}
\begin{list}{}{}
	\item  Proficient in English and Korean.
\end{list}

%\newpage

\vspace{0.1cm}
%\begin{multicols}{2} [\subsection*{REFERENCES}]

\subsection*{REFERENCES}
\hrule
\vspace{0.2cm}

\subsubsection*{FROM ACADEMIA}
\begin{footnotesize}

\begin{multicols}{2} 
\noindent 
Prof. Angelos D. Keromytis \\
Associate Professor \\
Dept. of Computer Science \\
1214 Amsterdam Ave. Mailcod 0401
New York NY 10027 \\
Phone: (212) 939-7095 \\
angelos@cs.columbia.edu \\

\noindent
Prof. Junfeng Yang\\
Assistant Professor \\
Dept. of Computer Science \\
1214 Amsterdam Ave. Mailcod 0401
New York NY 10027 \\
Phone: (212) 939-7012 \\
junfeng@cs.columbia.edu \\
\columnbreak

\noindent
Prof. Roxana Geambasu\\
Assistant Professor \\
Dept. of Computer Science \\
1214 Amsterdam Ave. Mailcod 0401
New York NY 10027 \\
Phone: (212) 939-7099 \\
roxana@cs.columbia.edu \\

\noindent
Prof. Georgios Portokalidis \\
Assistant Professor \\
Castle Point on Hudson \\
Hoboken, NJ 07030 \\
Phone: (201) 216-8249 \\
 georgios.portokalidis@stevens.edu \\
\end{multicols}

%\begin{tabbing}
%xxxx\=xxxxxxxxxxxxxxxxxxxxx\=xxxx\=xxxxxxxxxxxxxxxxxxxxxxxxxxxxxxxxxxxxxxxxx\=\kill
%\>{Nick McKeown}\> :\>nickm@stanford.edu\\
%\>{Balaji Prabhakar}\>:\>balaji@stanford.edu\\
%\>{Ajit Shelat}\>:\>ajit\_shelat@pmc-sierra.com\\
%\>{S.S.S.P. Rao}\>:\>ssspr@cse.iitb.ernet.in\\
%\end{tabbing}
%\newpage

\end{footnotesize}
\end{small}
\end{document}
%%%%%%%%%%%%%%%%%%%%%%%%%%%%%%%%%%%%%%%%%%%%%%%%%%%%%%%%%%%%%%%%%%%%%%%%
