\documentclass[letterpaper, 10pt]{article}
\topmargin-2.0cm

\usepackage{fancyhdr}
\usepackage{lastpage}
\usepackage{xspace}
\usepackage[usenames,dvipsnames,svgnames,table]{xcolor}

% Color Information from -
% http://www-h.eng.cam.ac.uk/help/tpl/textprocessing/latex_advanced/node13.html

\advance\oddsidemargin-1in
%\advance\evensidemargin-0.65in
\textheight9.2in
\textwidth6.9in
\newcommand\bb[1]{\mbox{\em #1}}
\def\baselinestretch{1.05}

\newcommand{\hsp}{\hspace*{\parindent}}
\definecolor{gray}{rgb}{0.4,0.4,0.4}

\begin{document}
\thispagestyle{fancy}

% Leave Left and Right Header empty.
\lhead{}
\rhead{}

\renewcommand{\headrulewidth}{0pt} \renewcommand{\footrulewidth}{0pt}
\fancyfoot[C]{\footnotesize
\textcolor{gray}{http://www.cs.columbia.edu/$\sim$jikk/application}}

\pagestyle{fancy}
\lhead{\textcolor{gray}{\it Kangkook Jee}}
\rhead{\textcolor{gray}{\thepage /\pageref{LastPage}}}

%Macros of my own
\def\libdft{libdft\xspace}
\def\TFAFull{Taint Flow Algebra\xspace}
\def\TFA{TFA\xspace}
\def\SR{ShadowReplica\xspace}
\def\ie{i.e.,\xspace}

\newif \ifcomments
%\commentstrue

\ifcomments
\newcommand{\jikk}[1]{{---\textcolor{red}{#1}---}}
\else
\newcommand{\jikk}[1]{}
\fi
%

% This kind of makes 10pt to 9 pt.
\begin{small}

%\vspace*{0.1cm}
\begin{center}
{\LARGE \bf RESEARCH STATEMENT}\\
\vspace*{0.1cm}
{\normalsize Kangkook Jee (jikk@cs.columbia.edu)}
\end{center}
%\vspace*{0.2cm}

%\begin{document}
%\centerline {\Large \bf Research Statement for Kangkook Jee}
%\vspace{0.5cm}

% Write about research interests...
%\footnotemark
%\footnotetext{Check This}

As the entire computing industry rapidly evolving, the task of building secure
and  reliable computing environment is getting harder and harder most notably
for failing to have security systems that satisfy the following properties
simultaneously.

\begin{itemize}

        \item {\bf Effectiveness} in detecting and preventing wide range of
                attack vectors immune to software failures caused either by
                active exploitation of malicious attackers or unintended
                discovery of developer bugs.

        \item {\bf Efficiency} by incurring minimal amount of overhead(e.g.,
                power, response time) for its deployment.

        \item {\bf Coverage} in providing end-to-end protection. \jikk{the
                entire software execution stack} 

\end{itemize}

Making the situation worse, unforeseeable nature of the computation model
evolution and user's response to it only exacerbate the problem as it re-defines
our notion about traditional computation model repeatedly.
%
Many industry vendors as well as research communities have driven the
innovations towards the secure system design and this is the space in which my
research interest is.

My main research contributions have been concretized over the past a few years
as I developed and improved well-known security system of data flow
tracking~(DFT) supporting software binaries for coverage.
%
To address the issue of high overhead, inherent to the systems of the same
kind, I leveraged a number of recent research innovations from different
software system layers. Specifically, I have been interested in combining
static and dynamic analysis to address the issue of high overhead.
%
Although my research have focused on a specific security architecture, I
believe that the insights that I learned can benefit the area of system
security in general since DFT share many architectural characteristics with
other security architectures.

In the next few paragraphs I describe my past works, current interest, and
future research directions in system security.

\subsection*{Data Flow Tracking(DFT) System Innovations}
% Paragraph 1: A brief paragraph sketching the overarching thematics and topic
% of your research, situating it disciplinary.
%
DFT has been my main research topic which I extensively explored various
aspects of the technology for its {\bf effectiveness} in serving many different
topics of research as well as for its {\bf efficiency} in making it more
adopted by production systems.
%
Typical DFT system is composed of three different operations {\it i)} {\it
tagging} input variables come in though any untrusted sources {\it ii)} {\it
propagating} tag values associated to the variables as the program executed
{\it iii)} {\it checking} for unintended usages of tagged variables. 
%
Our DFT implementations injects these operations using VM-based instrumentation
framework, which is PIN Dynamic Binary Instrumentation~(DBI) framework, to
apply the technology to COTS binaries pursuing the complete {\bf coverage} of
the entire execution stack.
%
Besides implementing DFT system that correctly operates, I invented number of
novel optimizations and architectures that alleviates its excessive slowdown
which is inherent to DFT implementations for COTS binaries and system security
tools of the same kind. And this is what I consider as the main contribution of
my research.

% libdft
\subsubsection*{\libdft: High performance fine-grained DFT system for COTS
software}

\libdft~\cite{libdft:2012vee} is our initial prototype implementation for DFT
that propagates tag values in byte granularity.
%
With its generic API, \libdft enables users to build tools that satisfy their
own demand. 
%
%It implements a process-level protection~(in contrast to system-wide) by
%injecting DFT logic for each {\it x86} instruction leveraging VM-based
%instrumentation framework.
%
Core to \libdft's performance improvement is in understanding the precise
definition of taint propagation semantics for {\it x86} instruction set as well
as the structural limitations of VM-based instrumentation layer. 
%
%\libdft addresses these issues by introducing optimal tracking codes
%customized for different types of {\it x86} instructions. 
%
Two main considerations that we made for the implementation are {\it i)}
minimization of the cost required for context switching between the original
application and DFT analysis contexts and {\it ii)} efficient design and
management of shadow memory space.
%
Evaluation for the prototype shows $\sim$10$\times$ - $\sim$11$\times$ slowdown
over the native execution for a number of standard benchmark suites and this
was a significant performance improvement over previous DFT implementations by
the time we introduced the system.\jikk{more justification needed for this
statement}

% TFA (Taint Flow Algebra)
\subsubsection*{\TFA: A framework for principled optimizations for DFT}

Even with the highly crafted/customized tracking logics that we invented for
\libdft, its operation still remained sub-optimal for not being able to address
the fundamental limitations common to up-to-date DFT implementations. 
%
These are about lacking in understandings for {\it i)} the global context of
DFT operations and {\it ii)} the semantics of DFT operations which is clearly
different from that of the original program execution.
%
\TFAFull~(\TFA)~\cite{tfa:2012ndss} is a special purpose Intermediate
Representation~(IR) that captures DFT logics extracted from the original {\it
x86} binary. With this representation, we apply number of traditional compiler
optimizations as well as DFT specific optimizations that we developed to
overcome aforementioned limitations.
%
For this, we developed the framework composed of two sub-components. The first
one is the static analysis component run from off-line analysis phase and
prepares the optimized DFT logics. The second one is the dynamic runtime
component that enforces the optimized DFT operations back to the original
program at runtime.
%
% may want to mention about 'feedback loop' connecting two components.
%
As a result, TFA could achieve \(\sim\)~2\(\times\) performance improvement
over  our baseline tool(\libdft) when it is evaluated against the same set of
standard benchmark suites without compromising the correctness~\footnote{In
        this context, by {\it correctness}, we aim to ensure the same level of
security guarantees that the baseline tool (\libdft) can provide.} of DFT
operations.

% ShadowReplica
\subsubsection*{\SR: A framework for parallelized execution of DFT}
%
Given that \TFA represents DFT logics extracted from the original program and
apply number of optimization approaches, \SR~\cite{sreplica:2013ccs} project
extends \TFA approach further to have more performance gain by running the
original program and the extracted DFT logics from two separate cores~(CPUs).
%
%Of course, we are not the first one who are aligned toward the approach of
%parallelized DFT analysis, but the most of previous proposals failed to
%achieve expected performance improvement mainly due to the high communication
%overhead connecting two different execution contexts. In most cases, the cost
%overwhelms the parallelization benefit by causing to require excessive amount
%of resources to make the system suitable for online deployment.
%
The main contribution of \SR project lies in addressing the issue of high
communication overhead and proposing a framework  to defend against any
software failures/exploitations at runtime. Different from previous proposals,
the system only requires a single additional analysis thread per an application
thread and the latter~(one that executes optimized \TFA representation) runs at
the same or faster speed than that of the former making synchronization between
two different contexts easier.
%
In \SR project, we again dedicate static analysis phase {\it i)} to have DFT
logic extracted from the target application binary and represent them in \TFA
\xspace{\it ii)} to minimize the volume and frequency of the communication
requirement.
%
Architectural consideration is required in designing the communication channel,
otherwise the channel would become a performance bottleneck as it fails to
schedule two threads for their shared resources of CPU caches.
%
As a result, \SR framework could improve the overhead by $\sim$2$\times$,
$\sim$4$\times$ over \TFA and \libdft respectively when it is evaluated for the
same standard benchmark suites, achieving $\sim$2.75$\times$ slowdown over the
native execution.

Interestingly, although \SR takes up additional CPUs to achieve better
performance, the entire framework consumes the same or less amount of CPU
cycles than that of comparable in-lined DFT implementation~(in this case \TFA)
for the same analysis.
%
In this case, the benefit of having smaller instrumented code for event
collection~(when it is compared to the size of code of in-lined DFT analysis)
and its subsequent performance saving surpassed the extra CPU cycles paid to
establish the communication channel and exchange data between two threads.

\subsection*{Current research: Practical and accurate security architecture}
%
My main research interest lies in the area of the system security in general
that serves for the safety and reliability of software systems and I could have
accumulated in depth insights and experiences about number of technologies
throughout my research career.
%
The followings are research topics that I am currently working on or keeping as
short-term research topics which would help readers to have better
understanding about my research direction.

\subsubsection*{A comparison study of DBI frameworks for security applications}
%
Oftentimes, security researchers choose one of the following three Dynamic
Binary Instrumentation~(DBI) frameworks -- PIN, DynamoRIO, and Valgrind -- as
their instrumentation medium when they need to in-line their security logic to
COTS binaries.
%
In this project, we carry out a comparison study that investigates various
aspects of these frameworks. As their behavioral and performance implications
significantly differ by each one's design philosophy and intended problem
domain, we propose number of criteria~(e.g., Userability,
Effectiveness/capability, efficiency) to empirically evaluate these frameworks.
%
The goal of the project is to provide a guideline for users who want to find a
instrumentation framework that meets for their demand.

\subsubsection*{General architecture of parallelized analysis} 
%
DFT is an instance of security analysis that resort to VM-based instrumentation
to in-line its logic to COTS binaries and suffers from high overhead.
%
Currently, we are working on a project that extends \SR approach to define
common API support other analyses expecting to have similar performance gain as
DFT did. 
%
As a first step, the project will begin to support other analyses of Control
Flow Integrity~(CFI) and memory integrity tool~(similar to Memcheck) then we
will look for more security analyses.

\subsubsection*{Further performance acceleration approaches}
%
Even with the substantial amount of performance improvement that our research
could make for DFT systems, I should admit that we have not yet reached the
level where industry and research community would accept as adoptable to their
productions.
% - the following may be redundant.
%And there is a consensus around research communities that whatever benefit the
%security system would bring to its users, $5\%$ overhead is a threshold that
%would prevent the system being adapted as a production
%system.~\cite{ccs2013:invited_talk}
%
For DFT implementation and for most of similar security systems that in-line
monitoring logics to COTS binary, I could pin down two major overhead sources;
{\it i)} the cost for VM-based instrumentation and {\it ii)} the cost for
shadow memory management~(mainly occurs as it translates real address to it
counterpart shadow address).
%
Currently, I am working on projects of two different kinds to address these
issues. The first one resorts to additional hardware component and the second
one proposes a new hybrid instrumentation framework, a software only solution
that combines source-based and binary-only instrumentation.

{\bf Hardware-assisted \SR} 
%
%Although \SR minimizes the size of the code instrumented, the cost for
%original application execution along with instrumented code for event
%collection still takes up the large portion of the system's overall slowdown.
%Second to this, the cost for shadow memory translation also incur
%non-negligible amount of computations.
%
We are working on a system that introduces another pipeline stage to the
existing CPU architecture that records required information and transfer it to
another core. This would eventually replace the in-lined event collector and
connecting communication channel of \SR. Another H/W component we are
implementing is a new instruction that would assist real-to-shadow address
translation. The important design criteria for this project is to keep
additional hardware component minimal and flexible so as it can support other
kinds of analysis with simple modification of software component.

{\bf Hybrid instrumentation framework} This is a software only approach that
proposes a new instrumentation framework that maximize the benefit of
available~(partial) source access still supporting COTS binaries.
%
This combines two different instrumentation approaches into a single framework.
These are compiler-based~(LLVM) and VM-based~(PIN DBI) instrumentation
frameworks to support source-based and binary-only instrumentation respectively.
%
This hybrid approach introduces number of interesting research challenges that
include how to maintain a single execution unit integrating two very different
instrumentation approaches and support uniform API interface to support
different analyses.
%
%Intuition behind this idea is that in most cases, we have partial accesses to
%source base feasible for compiler-based instrumentation known to be several
%times faster than VM-based instrumentation, and for the rest of execution
%environments(which we do not have source access), we can use VM-based
%instrumentation to maintain completeness.

\subsubsection*{Evaluation framework the soundness of DFT analyses}
%
Although DFT has gained many attentions from research communities for its known
effectiveness, its {\it soundness} issue in terms of correct information flow
has not been explored thoroughly and the question about the existence and
frequency of incorrect flows~(false positives or false negatives) remains
unanswered. From this work, we introduce a new methodology to evaluate the
soundness of DFT systems and build a prototype framework implementation to
apply it.
%
The entire framework composed of two sub-components of {\it i)} taintedness
measurement component that evaluates different information flows and {\it ii)}
the input generation system that guides the measurement component leveraging
the approach of symbolic execution.\jikk{some more texts needed.}
%
In this project, we are planning to apply the methodology to evaluate number of
different DFT systems implemented for Android operating system.

%
% What am I going to do for next 10, 20 years.  Discuss about my research
% vision 
%
% ain't just enumerating another research interests?
\subsection*{Research agenda}
%
Besides making improvements for the runtime security architectures~(i.e., DFT
and its same kind), I have broader interest in topics related to secure and
reliable software system design.
%
First of all, we should have better educational approaches to train computer
science professionals with proper/acceptable/correct coding practices.
%
It is also crucial to improve static analysis techniques such whitebox and
blackbox software fuzzing to assist developers producing better programs. 
%
In my view, the goal of building secure system can be attained not by any
single research direction but by holistic consideration of all aforementioned
topics and some others.

My research direction will move towards building secure system by  maximizing
the benefit of the static analysis techniques to improve the dynamic defense
measures. 
%
The other way around, runtime inputs from dynamic component would guides to the
static counterpart to achieve better coverage and accuracy.
%
Complementing one another, I am confident that we are getting closer to the
goal of building secure and reliable system effective in countering wide range
of attack vectors~(often unpredictable) only requiring minimal amount of
overhead. Providing such service with complete coverage~(e.g., COTS binaries)
is another design focus.
%
From my research outputs thus far, I believe that I opened up the possibility
in this line of research.

%What is the contribution from this line of work -- high level perspectives
\subsection*{Summary}
In the course of my research, I could establish in depth experience and insight
about DFT systems which is hot system security topics that shares its core
characteristics with many other security system architectures.
%
In effort to make the technology closer to the production deployment, I also
invented number of novel optimization techniques and frameworks that combine
static and dynamic analysis approaches.
%
Besides DFT, I am also interested in number of system security topics related
to hardening software systems and make those immune to exploitable defects or
unintended bugs. 
%
My future research agenda mainly focuses on developing and improving system
security architectures that would harden softwares for it safety and
reliability. 
%
\end{small} 

\newpage

%refer to http://sites.stat.psu.edu/~surajit/present/bib.htm
\bibliographystyle{plain}
\bibliography{res}
\end{document}
