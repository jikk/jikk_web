
\documentclass[letterpaper, 10pt]{article}
\topmargin-2.0cm

\usepackage{fancyhdr}
\usepackage{lastpage}
\usepackage[dvips]{color}

% Color Information from -
% http://www-h.eng.cam.ac.uk/help/tpl/textprocessing/latex_advanced/node13.html

\advance\oddsidemargin-0.65in
\advance\evensidemargin-1.5cm
\textheight9.2in
\textwidth6.75in
\newcommand\bb[1]{\mbox{\em #1}}
\def\baselinestretch{1.05}

\newcommand{\hsp}{\hspace*{\parindent}}
\definecolor{gray}{rgb}{0.4,0.4,0.4}

\begin{document}
\thispagestyle{fancy}

% Leave Left and Right Header empty.
\lhead{}
\rhead{}

\renewcommand{\headrulewidth}{0pt} \renewcommand{\footrulewidth}{0pt}
\fancyfoot[C]{\footnotesize
  \textcolor{gray}{http://www.cs.columbia.edu/$\sim$jikk/application}}

\pagestyle{fancy}
\lhead{\textcolor{gray}{\it Kangkook Jee}}
\rhead{\textcolor{gray}{\thepage /\pageref{LastPage}}}

% This kind of makes 10pt to 9 pt.
\begin{small}

%\vspace*{0.1cm}
\begin{center}
{\LARGE \bf RESEARCH STATEMENT}\\
\vspace*{0.1cm}
{\normalsize Kangkook Jee (jikk@cs.columbia.edu)}
\end{center}
%\vspace*{0.2cm}

%\begin{document}
%\centerline {\Large \bf Research Statement for Kangkook Jee}
%\vspace{0.5cm}

% Write about research interests...
%\footnotemark
%\footnotetext{Check This}

% Say that research work has been both theoretical and practical.
My main research interest is about protecting/hardening software systems
against malicious attempts from attackers or gaining more reliability against
unintended software bugs leveraging the latest research innovations. Recently,
these innovations are invented from both areas of static and dynamic program
analyses and I am interested in combining those two different research
directions so that can serve/complement one another making the technology more
practical in serving for many real world scenarios.

\subsection*{Background and Current Work}
% Paragraph 1: A brief paragraph sketching the overarching thematics and topic
% of your research, situating it disciplinarily.
%
In my research history, I extensively explored various aspects of Data Flow
Tracking(DFT) for its effectiveness in serving many different areas of research
as well as making it efficient so that it can more be adopted by users who are
in need to protect their systems.
%
To improve  performance of the technology, which incurs prohibitive amount
overhead especially when it is applied to binaries, I developed and utilized
number of static program analysis technologies so that we can make DFT suitable
in many real world scenarios.
%
For this, my interested research area includes instrumentation schemes
leveraging virtualization layer and static program analyses for many different
substrates which include compiler techniques for programming language
representations and binary analysis.
%
Besides DFT, I also explored other program analysis techniques that can be used
to strengthen software systems i) software fuzzing -- symbolic execution,
blackbox/whitebox fuzzing ii) integer error study.


% Para 2: A summary of the dissertation research. This may replicate to some
% extent the paragraph on the dissertation in the cover letter but it must have
% more detail about the methods, the theoretical foundations, and most of all,
% the core arguments.  Here, give a chapter summary, approximately one sentence
% per chapter.
\subsubsection*{Dissertation Research}
Along with line of my research topics of my journey about making DFT practical
in commodity applications by addressing the issue of high overheads.


% Para 3: A brief description of the contribution of the dissertation research
% to your field or fields, and a summary of publications associated with the
% dissertation research, including a plan for the book, if you are in a book
% field.
\subsubsection*{Contributions}

% Part 4: some other research areas that I'm interested in. 
\subsubsection*{Integer Error Handling}

\subsubsection*{Improving Dynamic Binary Instrumentation (DBI)}

\subsubsection*{Symbolic Execution}


\subsection*{Security System Architecture ---  A Research Agenda}
Even though my research focus in the past mostly leaned toward the topic of DFT
and related technologies, this contributions and limitations from my experience
can easily generalized to related defense/reliability technologies.

\begin{itemize}
  \item Hardware component
  \item Framework for hybrid instrumentation
  \item Correctness evaluation data flow tracking technologies
\end{itemize}


\end{small}

\end{document}

