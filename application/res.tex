
\documentclass[letterpaper, 10pt]{article}
\topmargin-2.0cm

\usepackage{fancyhdr}
\usepackage{lastpage}
\usepackage{xspace}
\usepackage[usenames,dvipsnames,svgnames,table]{xcolor}

% Color Information from - http://www-h.eng.cam.ac.uk/help/tpl/textprocessing/latex_advanced/node13.html

\advance\oddsidemargin-0.65in
\advance\evensidemargin-1.5cm
\textheight9.2in
\textwidth6.75in
\newcommand\bb[1]{\mbox{\em #1}}
\def\baselinestretch{1.05}

\newcommand{\hsp}{\hspace*{\parindent}}
\definecolor{gray}{rgb}{0.4,0.4,0.4}

\begin{document}
\thispagestyle{fancy}

% Leave Left and Right Header empty.
\lhead{}
\rhead{}

\renewcommand{\headrulewidth}{0pt} \renewcommand{\footrulewidth}{0pt}
\fancyfoot[C]{\footnotesize
  \textcolor{gray}{http://www.cs.columbia.edu/$\sim$jikk/application}}

\pagestyle{fancy}
\lhead{\textcolor{gray}{\it Kangkook Jee}}
\rhead{\textcolor{gray}{\thepage /\pageref{LastPage}}}

%Macros of my own
\def\libdft{libdft\xspace}
\def\TFA{TFA\xspace}
\def\SR{ShadowReplica\xspace}

\newif \ifcomments
\commentstrue

%\ifcomments
%\newcommand{\jikk}[1]{{---\textcolor{red}{#1}---}}
%\else
%\newcommand{\jikk}[1]{}
%\fi
%
\newcommand{\jikk}[1]{{---\textcolor{red}{#1}---}}


% This kind of makes 10pt to 9 pt.
\begin{small}

%\vspace*{0.1cm}
\begin{center}
{\LARGE \bf RESEARCH STATEMENT}\\
\vspace*{0.1cm}
{\normalsize Kangkook Jee (jikk@cs.columbia.edu)}
\end{center}
%\vspace*{0.2cm}

%\begin{document}
%\centerline {\Large \bf Research Statement for Kangkook Jee}
%\vspace{0.5cm}

% Write about research interests...
%\footnotemark
%\footnotetext{Check This}

% Say that research work has been both theoretical and practical.
My main research interest is about protecting/hardening software systems
against malicious attempts from attackers or gaining more reliability against
unintended software bugs leveraging the latest research innovations. Recently,
these innovations are invented from both areas of static and dynamic program
analyses and my research interest lies in combining those two different
research directions so that can serve/complement one another making the
technology more practical in serving for many real world scenarios.

\subsection*{Background and Current Work}
% Paragraph 1: A brief paragraph sketching the overarching thematics and topic
% of your research, situating it disciplinary.
%
In my research history, I extensively explored various aspects of Data Flow
Tracking(DFT) for its effectiveness in serving many different areas of
research~\cite{libdft, tfa, sr} as well as making it efficient so that it can
more be adopted by users who are in need to protect their systems.

%
To improve  performance of the technology, which incurs prohibitive amount
overhead especially when it is applied to binaries, I developed and utilized
number of static program analysis technologies so that we can make DFT suitable
in many real world scenarios.
% 
For this, my interested research area includes instrumentation schemes
leveraging virtualization layer and static program analyses for many different
substrates which include compiler techniques for programming language
representations and binary analysis.
%
Besides DFT, my research interest spans number of different topics which are
aligned to serve for software security and reliability.
%
\subsubsection*{DFT systems} 

% libdft
We implemented \libdft~\cite{libdft} which is highly optimized/efficient DFT
system which runs for binary program executed from commodity hardware system.
\libdft supports generic API that helps anyone to customize and build tools
that serve for different problem domains. 
%
\libdft is implemented via instrumenting tracking logics along with individual
{\tt x86} instructions using tools for VM-hypervisor layer. For \libdft, we
employed PIN dynamic binary instrumentation~(DBI) framework. High efficiency of
\libdft achieved by correctly understanding the structural limitations of
typical DBI instrumentations. Our implementation explicitly addressed this
issue with highly crafted/customized code that implements tracking logics that
avoids issues common to tools that leverages DBI frameworks which includes
EFLAG fill/re-fill, shadow memory management. Its benchmark result showed
$\sim$10$\times$ - $\sim$11$\times$  slowdown over the native execution when it
is evaluated against SPEC 2006 CPU benchmark suite which is a significant
performance improvement over previous DFT implementations.\jikk{Justification
needed for this statement}

% TFA (Taint Flow Algebra)
The immediate follow-up work for \libdft is a work~\cite{tfa} which decouples
tracking logics from the original program logics and represent tracking logic
as as a specially designed intermediate representation~(IR) of \TFA.  Number of
optimizations for compiler and DFT specific ones are applied to \TFA. Then the
optimization result is injected back to the original program. 
%
The entire system for \TFA optimization is composed of two different components
of the static analysis and dynamic runtime. We dedicate the off-line analysis
phase to perform static analysis to generate the optimized tracking code which
is applied to the dynamic runtime. As a result we could achieve
\(\sim\)~2\(\times\) performance improvement over our baseline tool \libdft
without compromising the correctness~\footnote{In this context, we {\it
correctness} mean that the same level of security guarantees as the baseline
tool which is \libdft} of DFT operations.

% ShadowReplica 
Regarding DFT logic and the original execution logic
decoupled/extracted/separated from our \TFA work, \SR attempts to execute each
from two different cores(CPUs) to gain even further performance improvements.
Oftentimes, the issue of synchronizing the original application thread and DFT
analyzer thread prevent the system from achieving expected performance
improvement. Subsequently, the system cannot be used for detection of malicious
activities at runtime.
%
The communication channel that connects two different context incurs
prohibitive amount of overhead up to the level where it overtakes the
parallelization benefits. Our major contribution lies in mitigating
communication overhead again combining the static and dynamic analysis to
address the issue.
%
\SR system again improves the overhead about  $\sim$2$\times$,  $\sim$4$\times$
over our previous DFT implementations of \TFA and \libdft respectively,
achieving  $\sim$2.75$\times$ slowdown over the native execution. 

%What is the contribution from this line of work.

\subsubsection*{Software fuzzing}
Symbolic execution, whitebox / blackbox fuzzing

\subsubsection*{Integer errors}
For programs written in C/C++ undefined behaviors

\subsubsection*{Dynamic Binary Instrumentation (DBI)}
required medium for inlining runtime monitoring 

% Para 2: A summary of the dissertation research. This may replicate to some
% extent the paragraph on the dissertation in the cover letter but it must have
% more detail about the methods, the theoretical foundations, and most of all,
% the core arguments.  Here, give a chapter summary, approximately one sentence
% per chapter.
\subsubsection*{}

% Para 3: A brief description of the contribution of the dissertation research
% to your field or fields, and a summary of publications associated with the
% dissertation research, including a plan for the book, if you are in a book
% field.
\subsubsection*{}


\subsection*{Security System Architecture ---  A Research Agenda}
Even though my research focus in the past mostly leaned toward the topic of DFT
and related technologies, this contributions and limitations from my experience
can easily generalized to related defense/reliability technologies.

\begin{itemize}
  \item Hardware component
  \item Framework for hybrid instrumentation
  \item Correctness evaluation data flow tracking technologies
\end{itemize}


\end{small}

\end{document}

