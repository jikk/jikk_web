% !TEX root = proposal.tex
\section{Introduction} 
\label{sec:intro}

Protecting software system is a challenging task. 
%
Attack vectors available to attackers are often unknown and this leads to
incapable of defending zero day attacks.
%
Developers make mistakes in their codes.

% Securing/positioning the document -- needs elaboration.
%
In respond to these problems, we have seen many proposals that
inject/instrument one or more type of monitoring logics~\cite{cfi, memcheck,
dft} against the program that we want to protect and let them run
simultaneously to defend against various types of unexpected behaviors.
%
Research has been exploring three different approaches to instrument/in-line
monitoring logics to implement in-line monitors.
%
Number of criteria to compare and evaluate different approaches. efficiency,
coverage, flexibility(general ?).

\begin{itemize}
%
    \item {\bf Source code based:} Leveraging different representations exposed
in the process of program build. Abstract Syntax Tree~(AST) or compiler
specific Intermediate Representations~(IR) can be an instrumentation target.
This approach is reasonably fast incurring about or less than $\sim$ 2$\times$
overhead, but it can encounter an issue related to code {\it coverage} as in
most case we only have a partial source access about program's execution
environment for having 3rd party libraries~(\ie libc) that come as binaries.
%
    \item {\bf Hardware assisted:} This approach leverages hardware add-ons to
instrument monitoring defense logics to a runtime execution and it is the most
efficient approach being able to achieve near-to-native performance. The
approach is not flexible as users cannot modify monitoring logics once it is
fixed/burned/encoded to hardware. This is also not general as we have not seen
any commodity vendors that implements this kind of in-line monitors.
%
    \item {\bf VM-based instrumentation:}  This approach leverages widely
adopted VM-hypervisor for instrumentation. As this approach overcomes the
coverage issue supporting unknown(COTS) binaries and users can modify and
update their monitoring logics so as to fulfill defense
requirements\jikk{awkward!}.

\end{itemize}
%
These approaches in general, being effective in defending software systems
against aforementioned threats, often suffer from excessive amount of overhead
incurred due to {\it i)} instrumentation cost {\it ii)} defense/monitoring
logic itself.


