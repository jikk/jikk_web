% !TEX root = proposal.tex
\section{Background and Related Work}
\label{sec:related}

In this section, I present related work and my previous research results
which are relevant to my thesis.

\subsection{Related Work}

\subsubsection{Inline Monitors}
\label{ssec:inline}

Program protection and profiling using inline monitors are dynamic analysis
approaches that execute specific analysis logics along with the application
process. Instance of the technology includes data flow tracking
(DFT)~\cite{DFT}, memory integrity checking~\cite{memcheck}, control flow
integrity~\cite{cfi}, method counting, call graph profiling and so on.
Given that the technology can be implemented interleaving analysis/monitoring
logics into the program execution, we can choose different instrumentation
targets either of source code or program binaries and use different
instrumentation mediums to implement inline monitors.

Employing source code based approach, we can use representations
exposed~\cite{AST,LLVM-IR} by compiler internals as instrumentation targets or
source-to-source approaches~\cite{txl, cil} to make changes to source code.
This approach comes in reasonable amount of overhead roughly around $2\times$
or less, but it is limited in completeness not being able to support COTS
binaries (\ie 3rd party libraries).  An alternative can be binary
implementation approaches either based on process-wide virtualization using
dynamic binary instrumentation(DBI)~\cite{PIN, dynamoRIO, valgrind} or
system-wide virtualization~\cite{qemu,xen}. These address coverage issue as
these can handle unknown program binaries. However, it comes with excessive
amount of overhead which vary from $\times 5 \sim \times 100$ based on analyses
and application domains. Hardware assisted implementation~\cite{HARD, lba} can
implement inline monitors with minimal amount of overhead less than 5\% at most
supporting full coverage, but the we have not yet seen this functionality
supported by major vendors with their commodity production.

\subsubsection{Data Flow Tracking (DFT)}

DFT is one of inline monitoring approaches that accurately tracks selected data
of interest, as they flow during program execution. Among other uses, DFT has
been employed to provide insight in the behavior of applications and systems,
and to assist in the identification of configuration errors. Most prominently,
it has been used in the security field to defend against various software
exploits~\cite{}, and to enforce information flow by monitoring and restricting
the use of sensitive data~\cite{}. For the former, the network is usually
defined as the source of interesting or “tainted” data, while the use of
tainted data is disallowed in certain program locations (\eg, in instructions
manipulating the control flow of programs, such as indirect branch instructions
and function calls). For the latter, the developer or the user is responsible
for specifying the data that needs to be tracked and the restrictions on their
use.  

\jikk{we may want to talk about details of DFT logics from here.}

\begin{table}[h]
\begin{tabular}{|l|l|}
\hline
{\bf Instruction} & {\bf Tag propagation rule} \\ \hline \hline
{\tt ALU-OP OP1 $\leftarrow$ OP2}  & {\tt t(op1) $\vert=$ t(op2)}\\ \hline
MOV OP1 \textless-  OP2                                                            & t(op1) = t(op2)      \\ \hline
LOAD OP1 \textless- {[}OP2{]}                                                      & t(OP1) = t(OP2)      \\ \hline
\end{tabular}
\end{table}

The specifics of DFT can vary significantly depending on ones goals,
performance considerations, and deployment platform. One possible
classification of existing mechanisms can be made based on the means by which
the tracking logic is augmented on regular program execution. As we discussed
from Section~\ref{ssec:inline}, DFT can be performed by inserting data tracking
logic statically during the compilation of software, or by performing source-
to-source code transformation~\cite{}. It can also be applied dynamically by
augmenting instrumentation code on existing binaries using dynamic binary
instrumentation (DBI)~\cite{}  or a modified virtual machine (VM)~\cite{}.
Finally, DFT can be also performed in hardware~\cite{}.

\subsubsection{Parallelized Analysis}

\subsection{Previous Research}
\subsubsection{\libdft}
\subsubsection{\tfa}
\subsubsection{\sreplica}
