% !TEX root = proposal.tex
\section{Hypothesis} \label{sec:hypo} 

In this section, I discuss about hypotheses that we want to verify and confirm
from my thesis.

\subsection{\sreplica Generalization}

Even though, our previous work most focuses on a specific instance of inline
monitor (\ie DFT), we strongly believe that the approaches can easily be
generalized to support analyses of other kinds only with minimal amount of
effort.  Tabel~\ref{tab:analyses} contains a list of well-known inline
monitors~\cite{CAB} echo one's goal is either to protect or profile application
processes dynamically. Column entries categorizes monitors according to its
structural properties. Column entry {\it Shadow Memory} denotes whether the
monitor requires shadow memory area to keep track of updates from real
execution context. {\it Data dependency} denotes whether the monitor's
operations depend on it own previous updates. We have to consider data
dependencies between updates if we want to parallel the analysis performed by
the monitor. {\it Check Operation} denotes whether the monitor examines for
specific system activities. For instance, DTA tool checks for every {\tt RET}
structions to see whether a taintedness value is used for its operand to
deviate executions. In contrast, {\it cache simulation} does not have such
operations.

\begin{table}[h]
\begin{tabular}{|c|c|c|c|}
\hline
Analyses & Shadow Memory & Data Dependency & Check Operation \\ 
\hline \hline
Data Flow Tracking (DFT) & O & O & O \\ \hline
Control Flow Integrity (CFI) & O & O & O \\ \hline
\specialcell{Memory Integrity Tool \\ (Memcheck)} & O & O & O \\ \hline
LockCheck & O & O & O \\ \hline
Method Counting & O & O & O \\ \hline
Call Graph Profiling & O & O & O \\ \hline
Path Profiling & O & O & O \\ \hline
Cache Simulation & O & O & O \\ \hline
\end{tabular}
\caption{ categorizes different inline analyses based on requiring {\it shadow
memory}, analysis' {\it dependency to its previous operations}, existence of
{\it checking/asserting operations}. \label{tab:analyses}}
\end{table}

Our goal here is to extend \sreplica approach to support most of moniors
presented from Table~\ref{tab:analyses}. Eventual output would be a framework
with support API that would help users to writes tools that serve for their
needs. However, due to time and resource constraints, we are planning to verify
to feasiblity of the concept by extending the current implementation of
\sreplica to support control flow integrity (CFI)~\cite{} and memory integrity
tool~\cite{}.

\subsubsection{Control Flow Integrity (CFI)} 

\subsubsection{Memory Integrity Tool (Memcheck)}

\subsection{Instrumentation Infrastructure Study}
\subsubsection{DBI Study}
\subsubsection{DTrace Instrumentation}
\subsubsection{Hardware Assisted Parallel Analysis}
\subsection{Overhead Modeling Framework}
