% !TEX root = proposal.tex
\section{Hypothesis} \label{sec:hypo} 

In this section, I discuss about hypotheses that we want to verify and confirm
from my thesis.

\subsection{\sreplica Generalization}

Even though, our previous work most focuses on one kind of inline monitor (\ie
DFT), we strongly believe that the approaches can easily be generalized to
support analyses of other kinds only with minimal amount of effort.
Tabel~\ref{tab:analyses} contains a list of well-known inline
monitors~\cite{CAB} whose purpose is either to protect or profile application
processes dynamically. Table~\ref{tab:analyses} also contains column entries to
categorize monitors according to its architectures. 

\begin{table}[h]
\begin{tabular}{|c|c|c|c|}
\hline
Analyses & Shadow Memory & Data Dependency & Check Operation \\ 
\hline \hline
Data Flow Tracking (DFT) & O & O & O \\ \hline
Control Flow Integrity (CFI) & O & O & O \\ \hline
\specialcell{Memory Integrity Tool \\ (Memcheck)} & O & O & O \\ \hline
LockCheck & O & O & O \\ \hline
Method Counting & O & O & O \\ \hline
Call Graph Profiling & O & O & O \\ \hline
Path Profiling & O & O & O \\ \hline
Cache Simulation & O & O & O \\ \hline
\end{tabular}
\caption{ categorizes different inline analyses based on requiring shadow
memory area, analysis dependency to its previous operations, existence of
checking/asserting operations. \label{tab:analyses}}
\end{table}

\subsubsection{Control Flow Integrity (CFI)}
\subsubsection{Memory Integrity Tool (Memcheck)}

\subsection{Instrumentation Infrastructure Study}
\subsubsection{DBI Study}
\subsubsection{DTrace Instrumentation}
\subsubsection{Hardware Assisted Parallel Analysis}
\subsection{Overhead Modeling Framework}
