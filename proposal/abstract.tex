% !TEX root = proposal.tex
\begin{abstract}

%While our everyday dependency to the computer-based services gets more and more
%significant, it is becoming more difficult to protect it from attackers
%motivated for increased revenue, and equipped with wide range of attack vectors
%for their malicious activities.

%
One way to harden the software system against failures, due to malicious
attempts from attackers or unexpected disclosure of developer bugs, is to
in-line security and monitoring logics to programs that we want to protect and
execute simultaneously.
%
Being effective in defending against these failures, the approaches based on
in-line monitors have suffered from an inherent issue of too high overhead
which subsequently hinders the approaches adaptation to production systems.

Addressing the issue, a line of approaches that parallelizes the original
execution and monitoring logic arose from the past proposals, but most of them
have failed to be a practical solution for many reasons, but most notably for
having too high communication cost connecting two distinct contexts.

From this thesis, I introduce a methodology that will improve the parallel
analysis by defining the minimal but sufficient amount of information needed
from the original execution to restore the correct monitoring context from
another thread,  
%
and the optimality of the approach will be confirmed by the formal analysis
based on the information theory. \jikk{should think about it. more carefully.}

From my past research experience, I implemented a prototype implementation of
the approach for DFT analysis, which leveraged the off-line static analysis and
in turn could achieve $\sim$ $\times$2.10 speed-up over the previous DFT
implementation requiring the less amount of CPU cycles.

Generalizing the approach, I will extend the approach so as it can benefit the
different type of analyses~(CFI, Memory Integrity Tool) to gain the similar
performance improvement as DFT did from my past research experience. 
%
For this, I will also establish a classification framework to identify in-line
analyses that can be benefited from the approach.

As a next step, we will explore design choices to improve the instrumentation
layer that in-lines event collection and communication logics taking advantage
of innovations from the proposed approach.
%
%For this, I will {\it i)} improve VM-based instrumentation which is currently
%employed {it ii)} propose a H/W based architecture that support the our
%approach to parallelize in-line analysis.  \\

\end{abstract}
