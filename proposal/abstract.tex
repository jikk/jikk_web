% !TEX root = proposal.tex
\begin{abstract}

%While our everyday dependency to the computer-based services gets more and more
%significant, it is becoming more difficult to protect it from attackers
%motivated for increased revenue, and equipped with wide range of attack vectors
%for their malicious activities.

%
A common way to harden software systems against failures due to malicious
attempts from attackers or unexpected disclosure of developer bugs, is to
in-line security and monitoring logics to be executed simultaneously.
%
Being effective in defending against these failures, the approaches based on
in-line monitors have suffered from inherent issue of too high overhead which
subsequently hinders the approach's production system adoptation.

Addressing the issue, a number of proposals for parallel execution arose from
the past proposals,
%that decouples two different contexts~(the one for the original execution and
%the other for monitoring logics) and let each run from different CPU cores
%arose from the past proposals, 
but most of them have failed to be a practical solution most notably for having
too high communication cost between two threads(contexts).
%
%Addressing the issue, I implemented a framework that parallelize  Data flow
%Tracking~(DFT) minimizing the amount of communcation required between the
%original and analysis thread.

From this document, I want to propose a methodology that defines the optimal
amount of information needed from the original execution required to restore
the correct monitoring context from another thread.
%
One intermediate result, I presented a system that parallelize DFT analysis,
which implements the methodology from offline static analysis which in turn
could achieve x2.75 slowdown over native execution when it is evaluated for
SPEC 2006 CPU benchmarks.
%
Generalizing the approach, I am planning to advance the research direction to
see how can we apply the approach so that dfferent type of ananalyses by
defining a guide and anc common API. 
%
As a next step, we will explore design choices available for
instrumentation/in-lining layer required for event collection and communcation.
This direction of research will i) improve VM-based instrumentation
which is currently employed ii) propose H/W component to support
parallelized analysis in broad/general sense.


%%
%While our everyday dependency to the computer-based services gets more and more
%significant, it is becoming more difficult to protect it from attackers
%motivated for increased revenue, and equipped with wide range of attack vectors
%for their malicious activities.
%%
%Number of different security systems have been proposed to defend software
%systems against such threats, but none has succeeded yet for a number of
%reasons. 
%% effectiveness issue
%Firstly, some security measures are not capable of protecting softwares against
%certain attack methods or being accurate in defining malicious activities.
%% efficiency issue
%Secondly, some other measures can counter most of attack vectors with
%reasonable accuracy but it incurs non-negligible amount of overhead and
%prevents wide adaptation. 
%
%In my past efforts to build a security system that address the aforementioned
%issues, I developed and improved a security system of DFT by leveraging
%VM-based instrumentation to in-line its monitoring logics. 
%%
%Even with the substantial amount of performance improvement my work could make
%for DFT system, I should admit that it has not yet reached to the point where
%industry and research community would consider it as adoptable to their
%production systems.
%
%In this document, I propose a number of research directions that would address
%aforementioned issues.
%%
%% by exploring various design choice from different layers of computer systems
%% and establishing a measurement framework that evaluates the accuracy of the
%% security system.
%%
%Our past innovations for DFT can be further enhanced leveraging opportunities
%available from different layers of the computer system. I will also verify that
%we can build an evaluation framework that would systematically infer invariants
%regarding correctness of DFT operations.
%%
%% filling gap between semantic understanding and instruction level
%% implementations.
%%
%Lastly, I will confirm that these experience can be generalized and extended to
%implement different security measures.
%%


\end{abstract}
