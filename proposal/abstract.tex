% !TEX root = proposal.tex
\begin{abstract}

While our everyday dependency to the computer-based services gets more and more
significant, it is becoming more difficult to protect it from attackers
motivated for increased revenue, and equipped with wide range of attack vectors
for their malicious activities.
%
A common way to harden software systems against these attempts is to in-line
security and monitoring logics and let them be executed simultaneously.
%
Being very effective in defending against malicous attempts or unexpected
disclosure of developer bugs, he issue of  high overhead is the most serious
issue regarding the approach as it hinders production system adoptation.

%From my past research experience, I specifically focused on this issue of
%excessive slowdown leveraing innovations from different CS fields --
%programming language theory, operating systems.

One most interesting/promising approach to be considered is to
parallelized/decoupled execution of monitoring logics along with the original
program to reduce the runtime interference/metigation to the original
application thread and achieve lower latency.
%
However, many past parallelization research have failed to be a practical
solution for number of reason but most notably not addressing to too high
communcation cost.
%
Addressing the issue, I implemented a framework that parallelize  Data flow
Tracking~(DFT) minimizing the amount of communcation required between the
original and analysis thread.

From this thesis proposal,  we want to propose a framework that first defines
the minimal and required amount of information needed from the original
application to be used from analysis thread to restore the entire monitoring
context. 
%
Then I want to confirm the idea by generalization  to implement other types of
analysis.
%
Furthermore, we will implement a couple of frameworks to be used to improve or
replace the current instrumentation layer. This way, we can confirm again that
the validity of hyperthosis being effective regardless of its instrumenation
layer.

%%
%While our everyday dependency to the computer-based services gets more and more
%significant, it is becoming more difficult to protect it from attackers
%motivated for increased revenue, and equipped with wide range of attack vectors
%for their malicious activities.
%%
%Number of different security systems have been proposed to defend software
%systems against such threats, but none has succeeded yet for a number of
%reasons. 
%% effectiveness issue
%Firstly, some security measures are not capable of protecting softwares against
%certain attack methods or being accurate in defining malicious activities.
%% efficiency issue
%Secondly, some other measures can counter most of attack vectors with
%reasonable accuracy but it incurs non-negligible amount of overhead and
%prevents wide adaptation. 
%
%In my past efforts to build a security system that address the aforementioned
%issues, I developed and improved a security system of DFT by leveraging
%VM-based instrumentation to in-line its monitoring logics. 
%%
%Even with the substantial amount of performance improvement my work could make
%for DFT system, I should admit that it has not yet reached to the point where
%industry and research community would consider it as adoptable to their
%production systems.
%
%In this document, I propose a number of research directions that would address
%aforementioned issues.
%%
%% by exploring various design choice from different layers of computer systems
%% and establishing a measurement framework that evaluates the accuracy of the
%% security system.
%%
%Our past innovations for DFT can be further enhanced leveraging opportunities
%available from different layers of the computer system. I will also verify that
%we can build an evaluation framework that would systematically infer invariants
%regarding correctness of DFT operations.
%%
%% filling gap between semantic understanding and instruction level
%% implementations.
%%
%Lastly, I will confirm that these experience can be generalized and extended to
%implement different security measures.
%%


\end{abstract}
