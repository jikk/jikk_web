% !TEX root = proposal.tex
\begin{abstract}

%While our everyday dependency to the computer-based services gets more and more
%significant, it is becoming more difficult to protect it from attackers
%motivated for increased revenue, and equipped with wide range of attack vectors
%for their malicious activities.

%
A promising way to harden the software system against failures, either due to
malicious attempts from attackers or unexpected disclosure of developer bugs,
is to in-line security and monitoring logic to the executing programs that we
aim to protect.
%
Although such techniques are usually effective, in-line monitors inherently
exhibit high overhead which subsequently hinders their adaptation to production
systems.

In our past research, we have addressed this issue of high overhead for data
flow tracking (DFT) technology which is one representative instance of the
in-line monitor. Most notably, our latest framework decouples analysis logic
from the application process and has two different contexts run in parallel.
Our framework takes advantage of recent multi-core innovations and achieves a
$\sim$ $\times$2.10 speed-up over the previous DFT implementation when
evaluated upon the SPEC 2006 CPU benchmark suite requiring a smaller number of
CPU cycles.
%
Contrary to other proposed parallel frameworks that suffer from excessive
amount of communication volume and frequency, the core of our proposed solution
lies in defining the minimal subset state -- {\it (a)} effective address (EA)
{\it (b)} control flow trace -- needed from the application process to be
monitored, and devising a number of novel optimization schemes.

Based on previous results, we want to propose two hypotheses in this document
to be examined and verified in the thesis.
%
Given that the most monitor logic can be reconstructed from another thread with
the subset state we defined for DFT, we first want to confirm that our
parallelization approach for DFT can be generalized to support different types
of in-line monitors expecting to have similar performance improvement.
%
Secondly, we want to examine and explore alternative choices for the
instrumentation layer. An expensive VM-based instrumentation framework is used
in our previous work to in-line event collector and communication stubs to the
application process and this in turn accounted the most for the slowdowns. We
want to confirm that we can have further speedup by substituting the framework
with more efficient instrumentation technologies.
%
To help verifying proposed hypotheses, we also want to establish an overhead
model framework. With this analytic framework, we expect to analyze and predict
performance changes in respond to the modifications that we would make to
our system prior to real execution. 

\end{abstract}
