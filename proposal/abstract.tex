% !TEX root = proposal.tex
\begin{abstract}

%While our everyday dependency to the computer-based services gets more and more
%significant, it is becoming more difficult to protect it from attackers
%motivated for increased revenue, and equipped with wide range of attack vectors
%for their malicious activities.

%
One way to harden the software system against failures, due to malicious
attempts from attackers or unexpected disclosure of developer bugs, is to
in-line security and monitoring logics to programs that we want to protect and
execute simultaneously.
%
Being effective in defending against these failures, the approaches based on
in-line monitors have suffered from an inherent issue of high overhead which
subsequently hinders the approaches adaptation to production systems.

In our past research, we have addressed this issue of high overhead for data
flow tracking (DFT) technology which is one representative instance of the
inline monitor. Most notably, our latest framework decouples analysis logics
from the application process and has two different contexts run in parallel
taking advantage of recent multi-core innovation achieving $\sim$ $\times$2.10
speed-up over the previous DFT implementation when it is evaluated for SPEC
2006 CPU benchmark suite requiring the less amount of CPU cycles.
%
Different from the most parallel frameworks proposed from other research which
have suffered from excessive amount of communication volume and frequency, our
main contribution lies in defining the minimal subset state --  {\it (a)}
effective address (EA) {\it (b)} control flow trace -- for the communication,
and devising number of novel optimization schemes.

Based on previous results, we want to propose three
hypotheses from the document to be verified from the thesis.
%
Firstly, we want to confirm that our parallelization approach for DFT can be
generalized to speedup different type of inline monitors expecting to have
similar amount performance improvement.
%
Secondly, we want to establish a modeling framework to make decision about each
monitor's fitness to our parallel approach in advance, prior to real execution.
%
Lastly, we want to explore and examine available alternatives for the
instrumentation framework that we have used to in-line event collector and
communication stubs to the application process. 
%
%For this, I will {\it i)} improve VM-based instrumentation which is currently
%employed {it ii)} propose a H/W based architecture that support the our
%approach to parallelize in-line analysis.  \\

\end{abstract}
